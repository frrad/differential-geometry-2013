\documentclass[10pt]{article} 

\makeatletter
\renewcommand\section{\@startsection{section}{1}{\z@}%
                                  {-3.5ex \@plus -1ex \@minus -.2ex}%
                                  {2.3ex \@plus.2ex}%
                                  {\normalfont\large\bfseries}}
\makeatother

\addtolength{\oddsidemargin}{-.875in}
\addtolength{\evensidemargin}{-.875in}
\addtolength{\textwidth}{1.75in}
\addtolength{\topmargin}{-.875in}
\addtolength{\textheight}{1.75in}

\usepackage{amssymb}
\usepackage{graphicx}
\usepackage{amsmath}
\usepackage{amsthm}
\usepackage{titlesec}
\usepackage{titling}

\setlength{\droptitle}{-6em}
\posttitle{\par\end{center}\vspace{-4.8em}}

\newcommand{\R}{{\ensuremath{\mathbb{R}}} }
\newcommand{\C}{{\ensuremath{\mathbb{C}}} }
\newcommand{\Z}{{\ensuremath{\mathbb{Z}}} }
\newcommand{\RP}{{\ensuremath{\mathbb{RP}}} }
\newcommand{\CP}{{\ensuremath{\mathbb{CP}}} }

\newcommand{\del}{{\ensuremath{\partial}} }
\newcommand{\xt}[2]{\ensuremath{\Gamma^{#1}_{#2}}}

\DeclareMathOperator{\ric}{Ric}
\DeclareMathOperator{\scal}{scal}
\DeclareMathOperator{\tr}{tr}

\newcommand{\hint}[1]{{\emph{Hint:} #1}} %This line shows hints
%\newcommand{\hint}[1]{} %Use this line to hide all hints
\newcommand{\note}[1]{{(\emph{Note:} #1)}} %This line shows notes
%\newcommand{\note}[1]{} %Use this line to hide all notes

\newcommand{\pets}[3]{{Petersen, Chapter #1, Exercise #2 on p. #3.}}
\newcommand{\advsection}{\addtocounter{section}{1} }

\usepackage{fancyhdr}
\pagestyle{fancyplain}
\renewcommand{\headrulewidth}{0pt}

\begin{document}

\lhead{Frederick Robinson}
\rhead{Differential Geometry: Homework 3}

\section{\pets{5}{1}{149}}

\section{\pets{5}{2}{149}}

\section{\pets{5}{8(c)}{149} (The definition of \emph{totally geodesic} is on p. 145)}

\section{Let $p$ be a point in a Riemannian manifold $(M, g)$ and $\sigma \subset T_p M$ a two-dimensional subspace. For small $r > 0$, let $\Sigma_\sigma \subset M$ be the (diffeomorphic) image of $B(0,r) \cap \sigma \subset T_pM$ under the exponential map exp$_p$. Show that the sectional curvature $\sec(\sigma)$ at $p$ (computed inside $M$) is equal to the sectional curvature (that is, Gaussian curvature) of the surface $\Sigma_\sigma$ at $p$, in the induced metric.}

\section{Let $SO(n)$ be the Lie group of orthogonal matrices of determinant 1. Equip $SO(n)$ with a bi-invariant Riemannian metric $g$ of volume one, as constructed in the previous homework. The tangent space $T_I SO(n)$ can be identified with the space $\mathfrak{so}(n)$ of skew-adjoint matrices. Show that the exponential map (with respect to $g$) $$\exp_I : \mathfrak{so}(n) \to SO(n)$$ coincides with the usual matrix exponentiation $A \to e^A$. \\\\ \hint{Feel free to use Proposition 12 on p.79 in Petersen's book. Compare also exercise 19 in Petersen, Chapter 5, p.151.}}

\section{\pets{6}{5}{184}}

\section{\pets{6}{8}{184} (See the definition of a \emph{Killing field} on p.23.)}

\section{\pets{6}{10}{184}}
























\end{document}
