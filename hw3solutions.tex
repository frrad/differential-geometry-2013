\documentclass[10pt]{article} 

\makeatletter
\renewcommand\section{\@startsection{section}{1}{\z@}%
                                  {-3.5ex \@plus -1ex \@minus -.2ex}%
                                  {2.3ex \@plus.2ex}%
                                  {\normalfont\large\bfseries}}
\makeatother

\addtolength{\oddsidemargin}{-.875in}
\addtolength{\evensidemargin}{-.875in}
\addtolength{\textwidth}{1.75in}
\addtolength{\topmargin}{-.875in}
\addtolength{\textheight}{1.75in}

\usepackage{amsmath, amssymb, amsthm, fancyhdr, graphicx, titlesec, titling}

\setlength{\droptitle}{-6em}
\posttitle{\par\end{center}\vspace{-4.8em}}

\newcommand{\C}{\ensuremath{\mathbb{C}}}
\newcommand{\CP}{\ensuremath{\mathbb{CP}}}
\newcommand{\R}{\ensuremath{\mathbb{R}}}
\newcommand{\RP}{\ensuremath{\mathbb{RP}}}
\newcommand{\Z}{\ensuremath{\mathbb{Z}}}

\newcommand{\del}{\ensuremath{\partial}}
\newcommand{\xt}[2]{\ensuremath{\Gamma^{#1}_{#2}}}

\DeclareMathOperator{\ric}{Ric}
\DeclareMathOperator{\scal}{scal}
\DeclareMathOperator{\tr}{tr}
\DeclareMathOperator{\2}{II}

\newcommand{\hint}[1]{{\emph{Hint:} #1}} %This line shows hints
%\newcommand{\hint}[1]{} %Use this line to hide all hints
\newcommand{\note}[1]{{(\emph{Note:} #1)}} %This line shows notes
%\newcommand{\note}[1]{} %Use this line to hide all notes

\newcommand{\pets}[3]{{Petersen, Chapter #1, Exercise #2 on p. #3.}}
\newcommand{\advsection}{\addtocounter{section}{1} }

\pagestyle{fancyplain}
\renewcommand{\headrulewidth}{0pt}

\begin{document}

\lhead{Frederick Robinson}
\rhead{Differential Geometry: Homework 3}

\section{Assume that $(M,g)$ has the property that all geodesics exist for a fixed time $\epsilon > 0$. Show that $(M,g)$ is geodesically complete.}

Fix $p\ \in M, v \in T_p M$. By assumption there exists a geodesic $\gamma : (- \epsilon/2 , \epsilon / 2 ) \to M$ with $\gamma(0) = p, \dot{\gamma}(0) = v$. However, taking $p' = \gamma(\epsilon/2 ), v' = \dot{\gamma} ( \epsilon / 2)$ we get another geodesic $\tilde{\gamma} :  (- \epsilon/2 , \epsilon / 2 ) \to M$. Since they coincide at $p', v'$ we can extend $\gamma : (-\epsilon / 2 , \epsilon) \to M$. Repeating this process we produce $\gamma : \R \to M$.

\section{A Riemannian manifold is said to be homogeneous if the isometry group acts transitively. Show that homogeneous manifolds are geodesically complete.}

\section{Let $N \subset (M,g)$ be a submanifold. Let $\nabla^N$ denote the connection on $N$ that comes from the metric induced by $g$. Define the second fundamental form of $N$ in $M$ by $$\2(X,Y) = \nabla_X^N Y - \nabla_X Y$$ Show that $\2 =0$ on $N$ iff $N$ is totally geodesic. (The definition of \emph{totally geodesic} is on p. 145)}

A submanifold $N \subset (M,g)$  is said to be \emph{totally geodesic} if for each $p \in N$ a neighborhood of $0 \in T_p N$ is mapped into $N$ via the exponential map $\exp_p$.

\section{Let $p$ be a point in a Riemannian manifold $(M, g)$ and $\sigma \subset T_p M$ a two-dimensional subspace. For small $r > 0$, let $\Sigma_\sigma \subset M$ be the (diffeomorphic) image of $B(0,r) \cap \sigma \subset T_pM$ under the exponential map exp$_p$. Show that the sectional curvature $\sec(\sigma)$ at $p$ (computed inside $M$) is equal to the sectional curvature (that is, Gaussian curvature) of the surface $\Sigma_\sigma$ at $p$, in the induced metric.}

\section{Let $SO(n)$ be the Lie group of orthogonal matrices of determinant 1. Equip $SO(n)$ with a bi-invariant Riemannian metric $g$ of volume one, as constructed in the previous homework. The tangent space $T_I SO(n)$ can be identified with the space $\mathfrak{so}(n)$ of skew-adjoint matrices. Show that the exponential map (with respect to $g$) $$\exp_I : \mathfrak{so}(n) \to SO(n)$$ coincides with the usual matrix exponentiation $A \to e^A$. \\\\ \hint{Feel free to use Proposition 12 on p.79 in Petersen's book. Compare also exercise 19 in Petersen, Chapter 5, p.151.}}

\section{Let $\gamma :[0,1] \to M$ be a geodesic. Show that $\exp_{\gamma(0)}$ has a critical point at $t \dot{\gamma}(0)$ iff there is a Jacobi field $J$ along $\gamma$ such that $J(0) = 0$, $\dot{J}(0) \neq 0$, and $J(t) =0$.}

\section{Let $\gamma$ be a geodesic and $X$ a Killing field in a Riemannian manifold. Show that the restriction of $X$ to $\gamma$ is a Jacobi field. (See the definition of a \emph{Killing field} on p.23.)}

\section{A Riemannian manifold is said to be $k$-\emph{point homogeneous} if for all pairs of points $(p_1, \dots, p_k)$ and $(q_1, \dots, q_k)$ with $d(p_i, p_j) = d(q_i, q_j)$ there is an isometry $F$ with $F(p_i) = q_i$. When $k=1$ we simply say tat the space is homogeneous.}
\subsection{Show that a homogeneous space has constant scalar curvature.}
\subsection{Show that if $k > 2$ and $(M,g)$ is $k$-point homogeneous, then $M$ is also $(k-1)$-point homogeneous.}
\subsection{Show that if $(M,g)$ is two-point homogeneous, then $(M,g)$ is an Einstein metric.}
\subsection{Show that if $(M,g)$ is three-point homogeneous, then $(M,g)$ has constant curvature.}
\subsection{Classify all three-point homogeneous spaces. \hint{The only one that isn't simply connected is the real projective space.}}



\end{document}