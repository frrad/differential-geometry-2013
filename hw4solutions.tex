\documentclass[10pt]{article} 

\makeatletter
\renewcommand\section{\@startsection{section}{1}{\z@}%
                                  {-3.5ex \@plus -1ex \@minus -.2ex}%
                                  {2.3ex \@plus.2ex}%
                                  {\normalfont\large\bfseries}}
\makeatother

\addtolength{\oddsidemargin}{-.875in}
\addtolength{\evensidemargin}{-.875in}
\addtolength{\textwidth}{1.75in}
\addtolength{\topmargin}{-.875in}
\addtolength{\textheight}{1.75in}

\usepackage{amsmath, amssymb, amsthm, fancyhdr, graphicx, titlesec, titling}

\setlength{\droptitle}{-6em}
\posttitle{\par\end{center}\vspace{-4.8em}}

\newcommand{\C}{\ensuremath{\mathbb{C}}}
\newcommand{\CP}{\ensuremath{\mathbb{CP}}}
\newcommand{\R}{\ensuremath{\mathbb{R}}}
\newcommand{\RP}{\ensuremath{\mathbb{RP}}}
\newcommand{\Z}{\ensuremath{\mathbb{Z}}}

\newcommand{\del}{\ensuremath{\partial}}
\newcommand{\xt}[2]{\ensuremath{\Gamma^{#1}_{#2}}}


\newcommand{\adjd}[1]{(-1)^{(n-#1)(#1+1)} * d^{n-#1-1} * }

\DeclareMathOperator{\ric}{Ric}
\DeclareMathOperator{\scal}{scal}
\DeclareMathOperator{\tr}{tr}
\DeclareMathOperator{\2}{II}

\newtheorem{fact}{Fact}


\newcommand{\hint}[1]{{\emph{Hint:} #1}} %This line shows hints
%\newcommand{\hint}[1]{} %Use this line to hide all hints
\newcommand{\note}[1]{{(\emph{Note:} #1)}} %This line shows notes
%\newcommand{\note}[1]{} %Use this line to hide all notes

\newcommand{\pets}[3]{{Petersen, Chapter #1, Exercise #2 on p. #3.}}
\newcommand{\advsection}{\addtocounter{section}{1} }

\pagestyle{fancyplain}
\renewcommand{\headrulewidth}{0pt}

\begin{document}

\lhead{Frederick Robinson}
\rhead{Differential Geometry: Homework 4}

\section{In dimensions $4n$ we have that the Hodge $*$ : $H^{2n}(M) \to H^{2n}(M)$ satisfies $**=I.$ The difference in the dimensions of the eigenspaces for $\pm 1$ is called the \emph{signature} of $M:$ $$\tau(M) = \sigma(M) = \dim( \ker (*-I) - \ker (*+I)).$$ One can show that this does not depend on the metric used to define $*$, by observing that it is the index of the symmetric bilinear map $$H^{2n}(M) \times H^{2n}(M) \to \R, $$ $$(\omega_1, \omega_2) \to \int \omega_1 \wedge \omega_2.$$ Recall that the index of a symmetric bilinear map is the difference between positive and negative diagonal elements when it has been put into diagonal form. In dimension 4 one can show that $$\sigma(M) = \frac{1}{12 \pi^2} \int_M \left(  \left| W^+ \right|^2 - \left| W^- \right|^2 \right).$$ Using the exercises from chapter 4, show that for an Einstein metric in dimension 4 we have $$\chi(M) \geq \frac{3}{2} \sigma(M),$$ with equality holding iff the metric is Ricci flat and $W^-=0.$ Conclude that not all four manifolds admit Einstein metrics. In higher dimensions there are no known obstructions to the existence of Einstein metrics. \hint{consider connected sums of $\CP^2$ with itself $k$ times.}}

From \pets{4}{9}{109} we have
\begin{align*}
\chi(M) &= \frac{1}{8 \pi^2} \int_M \left( |W^+|^2 + |W^-|^2 + \frac{\scal^2}{24} \right)\\
&=  \frac{3}{2} \left[ \frac{1}{12 \pi^2} \int_M \left( |W^+|^2 - |W^-|^2 \right)+ \frac{1}{12 \pi^2} \int_M \left(  \frac{\scal^2}{24} + 2 \cdot |W^-|^2 \right) \right] \\
&=  \frac{3}{2}  \sigma(M) + \frac{1}{8 \pi^2} \int_M \left(  \frac{\scal^2}{24} + 2 \cdot |W^-|^2 \right)   
\end{align*}
So clearly, $\chi(M) \geq \frac{3}{2}\sigma(M)$ with equality exactly when the metric is Ricci flat with  $W^- =0$.

Let $C_k = \overbrace{\CP^2 \# \cdots \# \CP^2}^{k\ \mathrm{times}}$ denote the connect sum of $k$ copies of $\CP^2$ as suggested in hint. Since $\chi(\CP^2) = 3$ and $\chi( X \# Y) = \chi(X) + \chi(Y) -2 $, in general $\chi(C_k) = k+2$.

Compute also $\sigma(\C_k) = k$ since by dimension constraints $\sigma(\CP^2) = 1$ and $\sigma(X \# Y) = \sigma(X) + \sigma(Y)$. So in this case the above inequality asserts that
\[\chi(C_k) = k+2 \geq \frac{3}{2} k = \frac{3}{2} \sigma(C_k) \iff k \leq 4.\]
so there can be no Einstein metric on $C_k$ for $k > 4$.

\section{Describe all harmonic forms on the following Riemannian manifolds:}
\subsection{The torus $T^n = S^1 \times \cdots \times S^1$ with the flat product metric;}
Recall,
\[H_{\mathrm{dR}}^{p}(T^n) \simeq \R^{n \choose p}.\]
Recall also (see for instance Jost 2.1.27) that in the Euclidean metric on $\R^n$ the Hodge Laplacian of a $p$-form $\omega = \omega_{i_1 \cdots i_p} dx^{i_1} \wedge \cdots \wedge dx^{i_p} $ is given by  
\[\Delta \omega = \delta d \omega + d \delta \omega = - \sum_{j=1}^n \frac{\del^2 \omega_{i_1 \cdots i_p}}{( \del x^j )^2} dx^{i_1} \wedge \cdots \wedge dx^{i_p}\]

Therefore, the harmonic $p$-forms on $T^n$ are exactly those which are harmonic in each coordinate. Since $T^n$  compact, all harmonic functions are constant. 


\subsection{The sphere $S^n$ with its round metric;}
Recall,
\[H_{\mathrm{dR}}^{k}(S^n) \simeq \begin{cases} \R & \mbox{if } k = 0, n \\ 0 & \mbox{otherwise } \end{cases}\]

By compactness of $S^n$, the harmonic 0-forms are just the constant functions. 
Furthermore, as $\Delta$ and $*$ commute (Exercise \ref{commute}), the harmonic $n$-forms are just $*\omega$ for $\omega$ constant 0-forms.

\subsection{\label{f-study}The complex projective space $\CP^n$ with the Fubini-Study metric.}
Recall,
\[H_{\mathrm{dR}}^k(\CP^n) \simeq \begin{cases} \R & k \mbox{ even},\ 0 \leq k \leq 2 n, \\ 0 & \mbox{otherwise} \end{cases}\]
For 0-forms we have the constant functions. 

Powers of the K\"{a}hler form $\omega$ are harmonic on a K\"{a}hler manifold.
\begin{proof}
If $\alpha$ is a  harmonic form then so is $\alpha \wedge \omega$ since this operation commutes with the Laplacian. Thus it suffices to show that the K\"{a}hler form  is harmonic

Take a local frame $X_1, Y_1, \dots, X_m , Y_m$ with $J(X_i) = Y_i$. In this frame $\omega = \sum X_i^\flat \wedge Y_i^\flat$. Furthermore
\[*\omega = \sum X_1^\flat \wedge Y_1^\flat \wedge \dots \wedge \hat X_i^\flat\wedge \hat Y_i^\flat \wedge \dots \wedge X_m^\flat \wedge Y_m^\flat = \frac{1}{(m-1)!}\omega^{m-1}\]
Thus $\omega$ co-closed. Since $\omega$ is a K\"{a}hler form it's automnatically closed and therefore harmonic.
\end{proof}



\subsection*{Feel free to use (without proof) the computation of the deRham cohomology groups of these spaces. Thus, in each case it suffices to exhibit a basis of harmonic forms of the required cardinality.  \hint{For \ref{f-study}, consider the powers of the K\"ahler form.}}

\section{Show that the Laplacian on forms commutes with the Hodge star operator.}
\label{commute}

First verify that we have  
\[* \delta^k d^k = d^{n-k-1} \delta^{n-k-1} * \iff * \delta^{n-k-1} d^{n-k-1} = d^{k} \delta^{k} * \]
\begin{align*}
    * \delta^k d^k & = * \adjd{k} d^k \\
    &=  (-1)^{(n-k)(k+1)} (* *) d^{n-k-1} * d^k\\
    &=  (-1)^{(n-k)(k+1)} (-1)^{k(n-k)} d^{n-k-1} * d^k\\
    &=  (-1)^{(n-k)(2k+1)}   d^{n-k-1} * d^k {** (-1)^{k(n-k)}}\\
    &=  (-1)^{(n-k)(3k+1)}   d^{n-k-1} * d^k * *  \\
    &=  (-1)^{(n-k)(3k+1)}   d^{n-k-1} (-1)^{(n-k)(k+1)}\adjd{(n-k-1)} *  \\
    &=  (-1)^{(n-k)(4k+2)}   d^{n-k-1} \delta^{n-k-1} *  \\
    &=  d^{n-k-1} \delta^{n-k-1} *. \\
\end{align*}

Consequently compute:
\begin{align*}
    * \Delta &= *d^{k-1} \delta^{k-1} + * \delta^k d^k\\
    &= \delta^{n-k}d^{n-k}* + d^{n-k-1}\delta^{n-k-1} * \\
    &= \Delta *
\end{align*}

\section{Consider the Laplacian $\Delta: \Omega^p(M) \to \Omega^p(M)$, where $M$ is a closed, oriented Riemannian manifold.}
Throughout we'll use  ``Fact 2'' from class:
\begin{fact} \label{fact2}
If $\{ \alpha_n \}$ is a sequence of smooth $p$-forms on $M$ such that $||\alpha_n || \leq c$ and $||\Delta \alpha_n || \leq c$ for all $n$ and some constant $c > 0 $ then $\{ \alpha_n \}$ has a Cauchy subsequence.
\end{fact}


\subsection{Prove that the eigenvalues of $\Delta$ are nonnegative, and have no finite accumulation points.}
Suppose $\lambda$ satisfies $\Delta \omega = \lambda \omega$ for some $\omega \in \Omega^p(M)$. Then $(\Delta \omega, \omega) = \lambda(\omega, \omega) \iff (\delta \omega, \delta \omega) + (d \omega, d \omega) = \lambda ||\omega||^2$. Therefore, $\lambda \geq 0$.

Suppose that $\alpha < \infty$ is an accumulation point of the eigenvalues of $\Delta$ and let $\{ \omega_i \}$ be an orthonormal sequence of eigenfunctions corresponding to distinct eigenvalues $\{\lambda_i\}$ converging to $\alpha$. (Note: such a sequence can be constructed by \ref{orthog}). Now, by Fact \ref{fact2} $\{\omega_i \}$ has a Cauchy subsequence, a contradiction.

\subsection{Prove that the eigenspaces of $\Delta$ are finite dimensional.}
Suppose that the eigenspace corresponding to $\lambda$ is infinite dimensional. Then by definition there exists an infinite sequence $\{ \omega_i \}$ of orthonormal eigenfunctions for $\lambda$. By Fact \ref{fact2} this basis has a Cauchy subsequence, a contradiction.

\subsection{Prove that the eigenspaces corresponding to distinct eigenvalues are orthogonal.}\label{orthog}
Let $A$ be a self-adjoint operator, $\alpha, \beta$ eigenvectors corresponding to eigenvalues $\lambda, \mu$. Then $(A \alpha, \beta) = \lambda ( \alpha, \beta)$ and $(\alpha, A \beta) = \mu (\alpha, \beta)$. Since $A$ is self-adjoint we have $\lambda(\alpha, \beta) = \mu(\alpha, \beta)$ so if $\lambda \neq \mu$ it must be that  $(\alpha, \beta)=0$ as desired. 

The Laplacian is self-adjoint.

\section*{\note{It can also be shown that $\Delta$ has infinitely many eigenvalues, and that the direct sum of all eigenspaces is dense in $\Omega^p(M)$. If you are interested, see Exercise 16 on p.254 in Warner, ``Foundations of Differentiable Manifolds and Lie Groups''}}

\section{Let $M$ be a closed, oriented, Riemannian 4-manifold. Let us identify $H^2(M; \R)$ with the space of harmonic 2-forms using the Hodge theorem. Note that the star operator $*$ acting on $\Omega^2(M; \R)$ satisfies $*^2 =1$, and therefore we have a direct sum decomposition $$\Omega^2(M; \R) = \Omega^+(M) \oplus \Omega^- (M),$$ where $\Omega^\pm$ are the eigenspaces of $*$ corresponding to the eigenvalues $\pm 1$. Restricting this decomposition to harmonic forms, we obtain another direct sum decomposition $$H^2(M; \R) = \mathcal{H}^+ \oplus \mathcal{H}^-,$$ where $\mathcal{H}^\pm$ are the eigenspace of $*$ acting on $H^2(M; \R)$, corresponding to eigenvalues $\pm 1$. \\\\ Let $d^+ : \Omega^1(M) \to \Omega^+(M)$ be the composition of $d$ with orthogonal projection to $\Omega^+$, and consider the three-term complex $$\Omega^0(M) \stackrel{d}{\to} \Omega^1(M) \stackrel{d^+}{\to} \Omega^+(M).$$ Show that the cohomology groups of this complex can be naturally identified with $H^0(M; \R)$, $H^1(M; \R)$, and $\mathcal{H}^+(M)$, respectively.}

\end{document}
