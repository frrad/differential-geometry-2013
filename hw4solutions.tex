\documentclass[10pt]{article} 

\makeatletter
\renewcommand\section{\@startsection{section}{1}{\z@}%
                                  {-3.5ex \@plus -1ex \@minus -.2ex}%
                                  {2.3ex \@plus.2ex}%
                                  {\normalfont\large\bfseries}}
\makeatother

\addtolength{\oddsidemargin}{-.875in}
\addtolength{\evensidemargin}{-.875in}
\addtolength{\textwidth}{1.75in}
\addtolength{\topmargin}{-.875in}
\addtolength{\textheight}{1.75in}

\usepackage{amsmath, amssymb, amsthm, fancyhdr, graphicx, titlesec, titling}

\setlength{\droptitle}{-6em}
\posttitle{\par\end{center}\vspace{-4.8em}}

\newcommand{\C}{\ensuremath{\mathbb{C}}}
\newcommand{\CP}{\ensuremath{\mathbb{CP}}}
\newcommand{\R}{\ensuremath{\mathbb{R}}}
\newcommand{\RP}{\ensuremath{\mathbb{RP}}}
\newcommand{\Z}{\ensuremath{\mathbb{Z}}}

\newcommand{\del}{\ensuremath{\partial}}
\newcommand{\xt}[2]{\ensuremath{\Gamma^{#1}_{#2}}}

\DeclareMathOperator{\ric}{Ric}
\DeclareMathOperator{\scal}{scal}
\DeclareMathOperator{\tr}{tr}
\DeclareMathOperator{\2}{II}

\newcommand{\hint}[1]{{\emph{Hint:} #1}} %This line shows hints
%\newcommand{\hint}[1]{} %Use this line to hide all hints
\newcommand{\note}[1]{{(\emph{Note:} #1)}} %This line shows notes
%\newcommand{\note}[1]{} %Use this line to hide all notes

\newcommand{\pets}[3]{{Petersen, Chapter #1, Exercise #2 on p. #3.}}
\newcommand{\advsection}{\addtocounter{section}{1} }

\pagestyle{fancyplain}
\renewcommand{\headrulewidth}{0pt}

\begin{document}

\lhead{Frederick Robinson}
\rhead{Differential Geometry: Homework 4}

\section{ \pets{7}{24}{233}}

\section{Describe all harmonic forms on the following Riemannian manifolds:}
\subsection{The torus $T^n = S^1 \times \cdots \times S^1$ with the flat product metric;}
\subsection{The sphere $S^n$ with its round metric;}
\subsection{\label{f-study}The complex projective space $\CP^n$ with the Fubini-Study metric.}
\subsection*{Feel free to use (without proof) the computation of the deRham cohomology groups of these spaces. Thus, in each case it suffices to exhibit a basis of harmonic forms of the required cardinality.  \hint{For \ref{f-study}, consider the powers of the K\"ahler form.}}

\section{Show that the Laplacian on forms commutes with the Hodge star operator.}

\section{Consider the Laplacian $\Delta: \Omega^p(M) \to \Omega^p(M)$, where $M$ is a closed, oriented Riemannian manifold.}
\subsection{Prove that the eigenvalues of $\Delta$ are nonnegative, and have no finite accumulation points.}
\subsection{Prove that the eigenspaces of $\Delta$ are finite dimensional.}
\subsection{Prove that the eigenspaces corresponding to distinct eigenvalues are orthogonal.}
\section*{\note{It can also be shown that $\Delta$ has infinitely many eigenvalues, and that the direct sum of all eigenspaces is dense in $\Omega^p(M)$. If you are interested, see Exercise 16 on p.254 in Warner, ``Foundations of Differentiable Manifolds and Lie Groups''}}

\section{Let $M$ be a closed, oriented, Riemannian 4-manifold. Let us identify $H^2(M; \R)$ with the space of harmonic 2-forms using the Hodge theorem. Note that the star operator $*$ acting on $\Omega^2(M; \R)$ satisfies $*^2 =1$, and therefore we have a direct sum decomposition $$\Omega^2(M; \R) = \Omega^+(M) \oplus \Omega^- (M),$$ where $\Omega^\pm$ are the eigenspaces of $*$ corresponding to the eigenvalues $\pm 1$. Restricting this decomposition to harmonic forms, we obtain another direct sum decomposition $$H^2(M; \R) = \mathcal{H}^+ \oplus \mathcal{H}^-,$$ where $\mathcal{H}^\pm$ are the eigenspace of $*$ acting on $H^2(M; \R)$, corresponding to eigenvalues $\pm 1$. \\\\ Let $d^+ : \Omega^1(M) \to \Omega^+(M)$ be the composition of $d$ with orthogonal projection to $\Omega^+$, and consider the three-term complex $$\Omega^0(M) \stackrel{d}{\to} \Omega^1(M) \stackrel{d^+}{\to} \Omega^+(M).$$ Show that the cohomology groups of this complex can be naturally identified with $H^0(M; \R)$, $H^1(M; \R)$, and $\mathcal{H}^+(M)$, respectively.}

\end{document}
