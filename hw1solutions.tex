\documentclass[10pt]{article} 



\makeatletter
\renewcommand\section{\@startsection{section}{1}{\z@}%
                                  {-3.5ex \@plus -1ex \@minus -.2ex}%
                                  {2.3ex \@plus.2ex}%
                                  {\normalfont\large\bfseries}}
\makeatother




\addtolength{\oddsidemargin}{-.875in}
\addtolength{\evensidemargin}{-.875in}
\addtolength{\textwidth}{1.75in}
\addtolength{\topmargin}{-.875in}
\addtolength{\textheight}{1.75in}


\usepackage{amssymb}
\usepackage{graphicx}
\usepackage{amsmath}
\usepackage{amsthm}
\usepackage{titlesec}

\usepackage{titling}
\setlength{\droptitle}{-6em}
\posttitle{\par\end{center}\vspace{-4.8em}}

\newcommand{\R}{{\ensuremath{\mathbb{R}}} }
\newcommand{\C}{{\ensuremath{\mathbb{C}}} }
\newcommand{\Z}{{\ensuremath{\mathbb{Z}}} }
\newcommand{\RP}{{\ensuremath{\mathbb{RP}}} }
\newcommand{\CP}{{\ensuremath{\mathbb{CP}}} }

\newcommand{\del}{{\ensuremath{\partial}} }
\newcommand{\xt}[2]{\ensuremath{\Gamma^{#1}_{#2}}}

\newcommand{\hint}[1]{{\emph{Hint:} #1}} %This line shows hints
%\newcommand{\hint}[1]{} %Use this line to hide all hints
\newcommand{\note}[1]{{(\emph{Note:} #1)}} %This line shows notes
%\newcommand{\note}[1]{} %Use this line to hide all notes

\newcommand{\pets}[3]{{Petersen, Chapter #1, Exercise #2 on p. #3}}
\newcommand{\advsection}{\addtocounter{section}{1} }


\usepackage{fancyhdr}
\pagestyle{fancyplain}
\renewcommand{\headrulewidth}{0pt}

\begin{document}

\lhead{Frederick Robinson}
\rhead{Differential Geometry: Homework 1}

\section{Let $G$ be a compact Lie group. Show that $G$ admits a bi-invariant metric, i.e., both right and left translations are isometries. \hint{Fix a left invariant metric $g_L$ and a volume form $\omega = \sigma^1 \wedge \cdots \wedge \sigma^n$ where $\sigma^i$ are left invariant 1-forms. Then define $g$ as the average over right translations: $$g(v,w) = \frac{1}{\int \omega} \int g_L (DR_x (v) , DR_x(w)) \omega.$$}}

Let $g_L$ be a left invariant metric on $G$, i.e. $g_L(v,w) = g_L(DL_x (v) , DL_x(w) ) $ for all $x \in G$. Suppose also $ \omega = \sigma^1 \wedge \cdots \wedge \sigma^n$ for  $\sigma^i$   left invariant 1-forms, and define as in the hint
\[g(v,w) = \frac{1}{\int \omega} \int g_L (DR_x (v) , DR_x(w)) \omega.\]

First compute
\begin{align*}g( DL_y(v),DL_y(w)) &= \frac{1}{\int \omega} \int g_L (DR_x (DL_y(v)) , DR_x(DL_y(w))) \omega \\ &= \frac{1}{\int \omega} \int g_L (DL_y(DR_x (v)) , DL_y(DR_x(w))) \omega\\ &= \frac{1}{\int \omega} \int g_L ( DR_x (v) ,DR_x(w)) \\&= g(v,w) \end{align*}
so $g$ left invariant.

From the right we have
\begin{align*}g( DR_y(v),DR_y(w)) &= \frac{1}{\int \omega} \int g_L (DR_x (DR_y(v)) , DR_x(DR_y(w))) \omega \\ &= \frac{1}{\int \omega} \int g_L (DR_{x \cdot y}(v) , DR_{x\cdot y}(w)) \omega\\ &= g(v,w)  \end{align*}


\section{Consider the upper-half plane $$\R^2_+  = \{ (x,y) \in \R^2 \mid y > 0 \}$$ with the hyperbolic metric $$\frac{dx^2 + dy^2}{y^2}.$$ Show that  the vertical line segment between $(0,1)$ and $(0,2)$ is the shortest path between these points.}

Let $\gamma(t) = \langle \gamma_x(t), \gamma_y(t) \rangle$ be a path $\gamma:[0,a]\to \R_+^2$ with $\gamma(0) = (0,1)$ and $\gamma(a) = (0,2)$. Assume without loss of generality that $\gamma$ is parameterized by (Euclidean) arclength. We can compute
\[|\gamma| = \int_0^a  \frac{1}{\gamma_y(t)} \left[ \left( \frac{\gamma'_x(t)}{1/\gamma_y(t) \left(  \gamma'_y(t)^2 + \gamma'_x(t)^2 \right)} \right) ^2 +  \left( \frac{\gamma'_y(t)}{1/\gamma_y(t) \left(  \gamma'_y(t)^2 + \gamma'_x(t)^2 \right)} \right) ^2 \right] dt \]
since we assumed $\gamma$ arclength parameterized this reduces to
\[|\gamma| = \int_0^a \gamma_y(t) dt.\]

By assumption, $\gamma'_y(t) \in [-1,1]$. In particular $\gamma'_y \leq 1 \Rightarrow \gamma_y(a-t) \geq 2 -t$. Notice $a \geq 1$ since $\gamma$ is arclength parameterized and the Euclidean distance between its endpoints is 1. Now compute:
\begin{align*}|\gamma| &=  \int_0^a \gamma_y(t) dt \\ & \geq \underbrace{\int_0^{a-1}\gamma_y(t) dt}_{(*)} + \int_1^0(  2-t ) dt  \end{align*}
Of course the contribution of $(*)$ is strictly positive if $a >1$ since $\gamma_y$ is restricted to take only positive values. Thus, our bound is tight if and only if $a = 1$ and $\gamma_y(a-t) = 2- t$, to wit, when
\[\gamma(t) =  \langle 0, 1 +t \rangle\]
as desired.

\section{Consider $\R_+^2$ with the hyperbolic metric as above. Let $v_0 = (0,1)$ be a tangent vector at the point $(0,1)$ of $\R_+^2$. Let $v(t)$ be the parallel transport of $v_0$ along the curve $x=t$, $y=1$. Show that $v(t)$ makes an angle $t$ with the direction of the $y$-axis, measured in the clockwise sense.}

First note that the nonzero components of the metric and respectively inverse metric are just
\[ g^{xx} = g^{yy} = y^2 \quad \mbox{and} \quad g_{xx} = g_{yy} = \frac{1}{y^2}.\]
We can therefore reduce the expression for Christoffel symbols
\[ \xt{m}{ij} = \frac{1}{2} \sum_k \left( \frac{\del}{\del x_i } g_{jk} + \frac{\del}{\del x_j}g_{ki} - \frac{\del}{\del x_k} g_{ij} \right) g^{km} = \frac{y^2}{2}  \left( \frac{\del}{\del x_i } g_{jm} + \frac{\del}{\del x_j}g_{mi} - \frac{\del}{\del x_m} g_{ij} \right)\]
Clearly this can only be nonzero if at least one of $i,j,m$ is $y$ and the others coincide. Thus compute:
\[\xt{y}{yy} =\xt{x}{xy} =\xt{x}{yx} = - \frac{1}{y}; \quad \xt{y}{xx} = \frac{1}{y}; \quad \xt{m}{ij} = 0   \mbox{ otherwise} .\]

Now let 
\[v(t) = \langle \sin t, \cos t \rangle \quad x(t) = \langle t, 1 \rangle \quad \dot{x}(t) =\langle 1 , 0 \rangle \]
and check
\[\frac{dv^x}{dt } + \sum_{i,j} \xt{x}{ij} v^j \frac{dx_i}{dt} = \cos t - \frac{1}{y} \cos t = 0 \quad \mbox{and} \quad\frac{dv^y}{dt } + \sum_{i,j} \xt{y}{ij} v^j \frac{dx_i}{dt} = - \sin t + \frac{1}{y} \sin t.\]

\section{For any $p \in (M,g)$ and orthonormal basis $e_1, \dots, e_n$ for $T_p M$, show that there is an orthonormal frame $E_1, \dots, E_n$ in a neighborhood of $p$ such that $E_i = e_i$ and $(\nabla E_i)|_p = 0 $. \hint{Fix an orthonormal frame $\overline{E}_i$ near $p \in M$ with $\overline{E}_i(p) = e_i$. If we define $E_i = \alpha_i^j \overline{E}_j$, where [$\alpha_i^j (x)$]$\in SO(n)$ and $\alpha_i^j(p) = \delta_i^j$, then this will yield the desired frame provided that the $D_{e_k} \alpha_i^j$ are appropriately prescribed. }}

\section{For any point $p$ in a  Riemannian manifold $(M,g)$, show that there exist coordinates $x^1, \dots, x^n$ near $p$ such that $ \del_i = e_i$ and $\nabla \del_i = 0 $ at $p$.}

\section{Let $(M,g)$ be oriented and define the Riemannian volume form $d$vol as follows: $$d\mbox{vol} ( v_1, \dots, v_n) = \det(g (v_i , e_j)),$$ where $e_1, \dots, e_n$ is a positively oriented orthonormal basis for $T_p M$.}

\section{\pets{2}{11}{58}}

\section{\pets{2}{13}{58}}

\end{document}
