\documentclass[10pt]{article} 

\makeatletter
\renewcommand\section{\@startsection{section}{1}{\z@}%
                                  {-3.5ex \@plus -1ex \@minus -.2ex}%
                                  {2.3ex \@plus.2ex}%
                                  {\normalfont\large\bfseries}}
\makeatother

\addtolength{\oddsidemargin}{-.875in}
\addtolength{\evensidemargin}{-.875in}
\addtolength{\textwidth}{1.75in}
\addtolength{\topmargin}{-.875in}
\addtolength{\textheight}{1.75in}

\usepackage{amssymb}
\usepackage{graphicx}
\usepackage{amsmath}
\usepackage{amsthm}
\usepackage{titlesec}

\usepackage{titling}
\setlength{\droptitle}{-6em}
\posttitle{\par\end{center}\vspace{-4.8em}}

\newcommand{\R}{{\ensuremath{\mathbb{R}}} }
\newcommand{\C}{{\ensuremath{\mathbb{C}}} }
\newcommand{\Z}{{\ensuremath{\mathbb{Z}}} }
\newcommand{\RP}{{\ensuremath{\mathbb{RP}}} }
\newcommand{\CP}{{\ensuremath{\mathbb{CP}}} }

\newcommand{\del}{{\ensuremath{\partial}} }
\newcommand{\xt}[2]{\ensuremath{\Gamma^{#1}_{#2}}}
\newcommand{\ric}{\mbox{Ric}}
\newcommand{\scal}{\mbox{scal}}

\newcommand{\hint}[1]{{\emph{Hint:} #1}} %This line shows hints
%\newcommand{\hint}[1]{} %Use this line to hide all hints
\newcommand{\note}[1]{{(\emph{Note:} #1)}} %This line shows notes
%\newcommand{\note}[1]{} %Use this line to hide all notes

\newcommand{\pets}[3]{{Petersen, Chapter #1, Exercise #2 on p. #3.}}
\newcommand{\advsection}{\addtocounter{section}{1} }

\usepackage{fancyhdr}
\pagestyle{fancyplain}
\renewcommand{\headrulewidth}{0pt}

\begin{document}

\lhead{Frederick Robinson}
\rhead{Differential Geometry: Homework 2}

\section{Give examples of Riemannian manifolds having:}
\subsection{positive scalar curvature but not positive Ricci curvature;}
\subsection{positive Ricci curvature but not positive sectional curvature;}
\subsection{positive sectional curvature but not positive curvature operator.}

\section{Show that a Riemannian manifold with parallel Ricci tensor has constant scalar curvature. In chapter 3 it will be shown that the converse is not true, and also that a metic with parallel curvature tensor doesn't have to be Einstein.}

\section{Let $G$ be a Lie group with a bi-invariant metric. Using left-invariant fields establish the following formuls. \hint{First go back to the exercises to chapter 1 and take a peek at chapter 3 where some of these things are proved.} }
\subsection*{(You can assume the following which are proved in Proposition 12 on p. 79. I suggest you read that proof.)$$\nabla_X Y = \frac{1}{2} [X,Y]. \quad R(X,Y)Z = \frac{1}{4}[Z,[X,Y]]. \quad g(R(X,Y)Z,W) = - \frac{1}{4} (g([X,Y],[Z,W]))$$}
\subsection{Show that the curvature operator is also nonnegative by showing that: $$g \left( \mathfrak{R}\left( \sum_{i=1}^k X_i \wedge Y_i \right) , \left( \sum_{i=1}^k X_i \wedge Y_i \right) \right) = \frac{1}{4} \left| \sum_{i=1}^k [X_i, Y_i ] \right|^2 .$$}
\subsection{Show that $\ric(X,X) =0$ iff $X$ commutes with all other left-invariant vector fields. Thus $G$ has positive Ricci curvature if the center of $G$ is discrete.}
\subsection{Consider the linear map $\Lambda^2 \mathfrak{g} \to [\mathfrak{g} , \mathfrak{g}]$ that sends $X \wedge Y$ to $[X,Y]$. Show that the sectional curvature is positive iff this map is an isomorphism. Conclude that this can only happen if $n=3$ and $\mathfrak{g} = \mathfrak{su} (2)$.}


\section{Consider a Riemannian metric $(M,g)$. Now \emph{scale} the metric by multiplying it by a number $\lambda^2$. Then we get  a new Riemannian manifold $(M, \lambda^2 g)$. Show that the new connection and $(1,3)$-curvature tensor remain the same, but that sec, scal, and $\mathfrak{R}$ all get multiplied by $\lambda^{-2}$.}

\section{Recall that complex manifolds have complex tangent spaces. Thus we can multiply vectors by $\sqrt {-1}$. As a generalization of this we can define an \emph{almost complex} structure. This is a $(1,1)$-tensor $J$ such that $J^2 =-I$. Show that the \emph{Nijenhuis tensor:} $$N(X,Y) = [J(X), J(Y)] - J([J(X),Y])- J([X,J(Y)]) - [X,Y]$$ is indeed a tensor. If $J$ comes from a complex structure then $N=0$, conversely Newlander \& Nirenberg have shown that $J$ comes from a complex structure if $N=0$.}

\section*{A \emph{Hermitian  structure} on a Riemannian manifold $(M,g)$ is an almost complex structure $J$ such that $$g(J(X), J(Y)) = g(X,Y).$$ The \emph{K\"ahler form} of a Hermitian structure is $$\omega(X,Y) = g(J(X),Y).$$ Show that $\omega$ is a 2-form. Show that $d \omega =0$ iff $\nabla J=0$. If the K\"ahler form is closed, then we call the metric a K\"ahler metric.}

\section{There is a strange curvature quantity we have not yet mentioned. Its definition is somewhat cumbersome and nonintuitive. First, for two symmetric $(0,2)$-tensors $h,k$ define the \emph{Kulkarni-Nomizu product} as the $(0,4)$-tensor $$h \circ k(v_1, v_2, v_3, v_4) = h(v_1,v_3) \cdot k(v_2, v_4) + h(v_2, v_4) \cdot k(v_1, v_3)$$ $$\hspace{3.65cm}- h(v_1, v_4) \cdot k(v_2, v_3) -h(v_2, v_3) \cdot k(v_1, v_4).$$ Note that $(M, g)$ has constant curvature $c$ iff the $(0,4)$-curvature tensor satisfies $R = c \cdot (g \circ g)$. If we use the $(0,2)$ form of the Ricci tensor, then we can decompose the $(0,4)$-curvature tensor as follows in dimensions $n \geq 4$ $$R = \frac{\scal}{2n ( n-1)} g \circ g + \left( \ric - \frac{\scal}{n} \cdot g \right) \circ g + W $$ When $n=3$ we have instead $$ R = \frac{\scal}{12} g \circ g + \left( \ric - \frac{\scal}{3} \cdot g \right) \circ g.$$ The $(0,4)$-tensor $W$ defined for $n > 3$ is called the \emph{Weyl tensor.}}
\subsection{Show that these decompositions are orthogonal, in particular: $$|R|^2 = \left| \frac{\scal}{2n ( n-1)} g \circ g \right| ^2  + \left| \left( \ric - \frac{\scal}{n} \cdot g \right) \circ g \right|^2 + |W|^2.$$}
\subsection{Show that if we conformally change the metric $\tilde{g} = f \cdot g$, then $\tilde{W} = f \cdot W$.}
\subsection{If $(M,g)$ has constant curvature, then $W=0$.}
\subsection{If $(M,g)$ is locally conformally equivalen to the Euclidean metric, i.e., locally we can always find coordinates where: $$g = f \cdot \left( (dx^1)^2 + \cdots + (dx^n)^2 \right), $$ then $W=0$. The converse is also true but much harder to prove.}



\section{Assume that we have a Riemannian immersion of an $n$-manifold into $\R^{n+1}$. If $n \geq 3$, then show that it can't have negative curvature. If $n=2$ give an example where it does have negative curvature.}

\section{Let $(M,g)$ be a closed Riemannian $n$-manifold, and suppose that there is a Riemannian embedding into $\R^{n+1}$. Show that there must be a point $p \in M$ where the curvature operator $\mathfrak{R}: \Lambda^2 T+p M \to \Lambda^2 T_p M$ is positive. \hint{Consider $f(x) = |x|^2$ and restrict it to $M$, then check what happens at a maximum.}}



\end{document}
